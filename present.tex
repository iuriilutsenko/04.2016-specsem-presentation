\documentclass[10pt,pdf,hyperref={unicode},envcountsect]{beamer}

\usetheme{Antibes}

\usepackage{hyperref}
\usepackage{cmap}				
\usepackage{mathtext} 			
\usepackage[T2A]{fontenc}		
\usepackage[utf8]{inputenc}		
\usepackage[english,russian]{babel}

%\usepackage[pdftex]{graphicx}
\usepackage{amsmath,amsfonts,amssymb,amsthm,mathtools}
\usepackage{icomma}
\usepackage{amsthm}
\usepackage{diagbox}

\mathtoolsset{showonlyrefs=true}

\usepackage{euscript}
\usepackage{mathrsfs}

\DeclareMathOperator{\sgn}{\mathop{sgn}}

\newcommand*{\hm}[1]{#1\nobreak\discretionary{}
{\hbox{$\mathsurround=0pt #1$}}{}}

\newtheorem{thm}{Теорема}
\newtheorem{defn}{Определение}

\title{Исследование схем правдоподобных рассуждений в системах искусственного интеллекта}
\author{Луценко Юрий Юрьевич}
\institute{МГУ им. М.В.Ломоносова \\
    \vspace{0.7cm}
    Научный руководитель: д.ф.-м.н. Соловьев С.Ю. \\
    \vspace{0.7cm}
}
\date{
    2016 г.
}

\setbeamertemplate{theorems}[numbered]
\setbeamertemplate{definitions}[numbered]

\makeatletter
\newenvironment<>{proofs}[1][\proofname]{%
    \par
    \def\insertproofname{#1\@addpunct{.}}%
    \usebeamertemplate{proof begin}#2}
  {\usebeamertemplate{proof end}}
\newenvironment<>{proofc}{%
  \setbeamertemplate{proof begin}{\begin{block}{}}
    \par
    \usebeamertemplate{proof begin}}
  {\usebeamertemplate{proof end}}
\newenvironment<>{proofe}{%
    \par
    \pushQED{\qed}
    \setbeamertemplate{proof begin}{\begin{block}{}}
    \usebeamertemplate{proof begin}}
  {\popQED\usebeamertemplate{proof end}}
\makeatother

\begin{document}
\begin{frame}
  \maketitle
\end{frame}

\section{Описание схемы Шортлиффа}

\begin{frame}
\frametitle{Схема Шортлиффа (MYCIN)}
\begin{itemize}
\item{Вычисляет, используя продукционные правила, суждения о гипотезах на основании суждений о фактах}
\item{Поддерживает ``интеллектуальный'' диалог экспертной системы с пользователем с целью получения оценочных суждений о фактах}
\item{Существенно использует модель вычисления так называемых коэффициентов уверенности
\begin{itemize}
\item{Коэффициенты уверенности правил - числа из полуинтервала $(0, 1]$}
\item{Коэффициенты уверенности фактов и гипотез - числа из отрезка $[-1, 1]$}
\end{itemize}
}
\end{itemize}
\end{frame}

\begin{frame}
\frametitle{Продукционные правила}
Будут рассмотрены правила вида
$$if\ Антецедент\ then\ Гипотеза\ with\ CF$$
\begin{itemize}
\item{Антецедент - формула, построенная из фактов и гипотез с помощью операций конъюнкции, дизъюнкции и отрицания}
\item{Гипотеза - одна из гипотез продукционной системы}
\item{CF - коэффициент уверенности правила}
\end{itemize}
\end{frame}

\begin{frame}
\frametitle{Операции и отношения над коэффициентами уверенности}
\begin{enumerate}[(i)]
\item{$min(a, b)$, $max(a, b)$}
\item{$not(a) = -a$}
\item{$rge(a) = \mathbb{I}(a \geqslant 0.2)$, $rle(a) = \mathbb{I}(a \leqslant -0.2)$}
\item{$tms(a, b) = a \times b$}
\item{$
  cmb(a, b) = \left\{
    \begin{array}{lr}
      a + b - a \cdot b,& a \geqslant 0, b \geqslant 0 \\
      \frac{a + b}{1 - min(|a|, |b|)},& a \cdot b < 0 \\ 
      a + b + a \cdot b,& a \leqslant 0, b \leqslant 0 \\
    \end{array}
  \right.
$}
\end{enumerate}
\end{frame}

\begin{frame}
\end{frame}

\section*{}
\begin{frame}[plain,c]
\begin{center}
\Huge Спасибо за внимание!
\end{center}
\end{frame}

\end{document} % конец документа