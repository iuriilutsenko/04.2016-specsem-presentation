\documentclass[10pt,pdf,hyperref={unicode},envcountsect]{beamer}

\usetheme{Antibes}

\usepackage{hyperref}
\usepackage{cmap}				
\usepackage{mathtext} 			
\usepackage[T2A]{fontenc}		
\usepackage[utf8]{inputenc}		
\usepackage[english,russian]{babel}

%\usepackage[pdftex]{graphicx}
\usepackage{amsmath,amsfonts,amssymb,amsthm,mathtools}
\usepackage{icomma}
\usepackage{amsthm}
\usepackage{diagbox}

\mathtoolsset{showonlyrefs=true}

\usepackage{euscript}
\usepackage{mathrsfs}

\DeclareMathOperator{\sgn}{\mathop{sgn}}

\newcommand*{\hm}[1]{#1\nobreak\discretionary{}
{\hbox{$\mathsurround=0pt #1$}}{}}

\newtheorem{thm}{Теорема}
\newtheorem{defn}{Определение}

\title{Исследование схем правдоподобных рассуждений в системах искусственного интеллекта}
\author{Луценко Юрий Юрьевич}
\institute{МГУ им. М.В.Ломоносова \\
    \vspace{0.7cm}
    Научный руководитель: д.ф.-м.н. Соловьев С.Ю. \\
    \vspace{0.7cm}
}
\date{
    2016 г.
}

\setbeamertemplate{theorems}[numbered]
\setbeamertemplate{definitions}[numbered]

\makeatletter
\newenvironment<>{proofs}[1][\proofname]{%
    \par
    \def\insertproofname{#1\@addpunct{.}}%
    \usebeamertemplate{proof begin}#2}
  {\usebeamertemplate{proof end}}
\newenvironment<>{proofc}{%
  \setbeamertemplate{proof begin}{\begin{block}{}}
    \par
    \usebeamertemplate{proof begin}}
  {\usebeamertemplate{proof end}}
\newenvironment<>{proofe}{%
    \par
    \pushQED{\qed}
    \setbeamertemplate{proof begin}{\begin{block}{}}
    \usebeamertemplate{proof begin}}
  {\popQED\usebeamertemplate{proof end}}
\makeatother

\begin{document}
\begin{frame}
  \maketitle
\end{frame}

\section{Описание схемы Шортлиффа}

\begin{frame}
\frametitle{Схема Шортлиффа (MYCIN)}
\begin{itemize}
\item{Вычисляет, используя продукционные правила, суждения о гипотезах на основании суждений о фактах}
\item{Поддерживает ``интеллектуальный'' диалог экспертной системы с пользователем с целью получения оценочных суждений о фактах}
\item{Существенно использует модель вычисления так называемых коэффициентов уверенности
\begin{itemize}
\item{Коэффициенты уверенности правил - числа из полуинтервала $(0, 1]$}
\item{Коэффициенты уверенности фактов и гипотез - числа из отрезка $[-1, 1]$}
\item{Важно выделить особые точки $\{-1, -0.2, 0, 0.2, 1\}$}
\end{itemize}
}
\end{itemize}
\end{frame}

\begin{frame}
\frametitle{Продукционные правила}
Используются правила вида
$$if\ Антецедент\ then\ Гипотеза\ with\ CF$$
\begin{itemize}
\item{Антецедент - формула, построенная из фактов и гипотез с помощью операций конъюнкции, дизъюнкции и отрицания}
\item{Гипотеза - одна из гипотез продукционной системы}
\item{CF - коэффициент уверенности правила}
\end{itemize}
\end{frame}

\begin{frame}
\frametitle{Операции и отношения над коэффициентами уверенности}
\begin{enumerate}[(i)]
\item{$min(a, b)$, $max(a, b)$}
\item{$not(a) = -a$}
\item{$rge(a) = \mathbb{I}(a \geqslant 0.2)$, $rle(a) = \mathbb{I}(a \leqslant -0.2)$}
\item{$tms(a, b) = a \cdot b$}
\item{$
  cmb(a, b) = \left\{
    \begin{array}{lr}
      a + b - a \cdot b,& a \geqslant 0, b \geqslant 0 \\
      \frac{a + b}{1 - min(|a|, |b|)},& a \cdot b < 0 \\ 
      a + b + a \cdot b,& a \leqslant 0, b \leqslant 0 \\
    \end{array}
  \right.
$}
\end{enumerate}
\end{frame}

\begin{frame}
\frametitle{Свойства функции комбинирования}
\begin{itemize}
\item{Область определения $D_{cmb} = [-1, 1] \setminus \{(-1, 1), (1, -1)\}$}
\item{$cmb(a, b) = cmb(b, a)$}
\item{$cmb(a, cmb(b, c)) = cmb(cmb(a, b), c)$}
\item{$cmb(a, b) = -cmb(-a, -b)$}
\item{$cmb(a, b) \in C(D_{cmb})$}
\item{$\forall{l}\ cmb(a, l) \leqslant cmb(b, l), a \leqslant b$}
\item{$-1 \leqslant cmb(a, b) \leqslant 1$}
\item{$cmb(a, -a) = 0$}
\item{$cmb(a, 0) = a$}
\item{$cmb(1, 1) = 1, cmb(-1, -1) = -1$}
\end{itemize}
\end{frame}

\section{Трансформации схемы Шортлиффа}
\begin{frame}
\frametitle{Трансформации схемы Шортлиффа}
Идея трансформации схемы Шортлиффа заключается в том, чтобы преобразовать имеющиеся операции таким образом, что появляется возможность оперировать с множеством $A$, отличным от отрезка $[-1, 1]$.

Для этого рассмотрим взаимно однозначную монотонно возрастающую функцию $h: [-1, 1] \rightarrow A$.
\end{frame}

\subsection{Семейство трансформации №1}
\begin{frame}
\frametitle{Семейство трансформаций №1}
Трансформации из семейства $G_1(\alpha)$, где $\alpha > 0$ переводят комбинирование в умножение, а умножение - в его изоморфный образ. Каждый элемент этого семейства - функция $g_1:[-1, 1] \rightarrow [0, +\infty]$.

\[
g_1(x)=\left\{
    \begin{array}{lr}
      (1 + x)^{\alpha},& -1 \leqslant x < 0 \\
      (1 - x)^{-\alpha},& 0 \leqslant x < 1 \\
      +\infty,& x = 1
    \end{array}
  \right.
\]

Обратная функция $g_1^{-1}: [0, +\infty] \rightarrow [-1, 1]$ выглядит следующим образом:
\[
g_1^{-1}(x)=\left\{
    \begin{array}{lr}
      x^{\frac{1}{\alpha}} - 1,& 0 \leqslant x < 1 \\
      1 - x^{-\frac{1}{\alpha}},& 1 \leqslant x < +\infty \\
      1,& x = +\infty
    \end{array}
  \right.
\]
\end{frame}

\subsection{Семейство трансформации №2}
\begin{frame}
\frametitle{Семейство трансформаций №2}
Главная особенность функции $h_1:[-1, 1] \rightarrow [-\infty, +\infty]$ из семейства трансформации $H_1(\beta)$ при $\beta > 1$ заключается в переводе функции комбинации в сумму.

\[
h_1(x)=\left\{
    \begin{array}{lr}
      -\infty,& x = -1 \\
      +\log_{\beta}(1 + x),& -1 < x < 0 \\
      -\log_{\beta}(1 - x),& 0 \leqslant x < 1 \\
      +\infty,& x = 1
    \end{array}
  \right.
\]

Ей соответствует обратная функция $h_1^{-1}: [-\infty, +\infty] \rightarrow [-1, 1]$:
\[
h_1^{-1}(x)=\left\{
    \begin{array}{lr}
      -1,& x = -\infty \\
      \beta^{x} - 1,& -\infty < x < 0 \\
      1 - \beta^{-x},& 0 \leqslant x < +\infty \\
      1,& x = +\infty \\
    \end{array}
  \right.
\]
\end{frame}

\section{Задача выявления и доопределения}
\begin{frame}
\frametitle{Задача выявления и доопределения}
\begin{enumerate}
\item{Задача обнаружения отображений}
\item{Задача подтверждения формул}
\item{Задача конструирования формул}
\item{Задача выявления отображений}
\item{Задача доопределения функции комбинирования}
\end{enumerate}
\end{frame}

\begin{frame}
\frametitle{Функция комбинирования Хамахера}
  Можно вывести и доопределить функцию комбинирования с использованием функции из семейства триангуляционных конорм Хамахера вида $\phi(a, b) = \frac{a + b + (r - 2)ab}{1 + (r - 1)ab}, r > 0$. В этом случае отображение принимает следующий вид:
  \[
h_r(x)=\left\{
    \begin{array}{lr}
      -\infty,& x = -1 \\
      -\log_{\beta}(\frac{r}{1 + x} + 1 - r),& -1 < x < 0 \\
      +\log_{\beta}(\frac{r}{1 - x} + 1 - r),& 0 \leqslant x < 1 \\
      +\infty,& x = 1
    \end{array}
  \right.
\]
Обратное отображение выглядит так:
\[
h_r^{-1}(x)=\left\{
    \begin{array}{lr}
      -1,& x = -\infty \\
      \frac{\beta^{x} - 1}{1+(r-1)\beta^{x}},& -\infty < x < 0 \\
      \frac{\beta^{x} - 1}{\beta^{x} + r - 1},& 0 \leqslant x < +\infty \\
      1,& x = +\infty \\
    \end{array}
  \right.
\]
\end{frame}

\begin{frame}
  Используя обозначенное выше отображение, можно вывести функцию комбинирования:
  \[
    cmb_r(a, b) = \left\{
      \begin{array}{lr}
        \frac{a + b - (r-2)ab}{1 + (r-1)ab},& a \geqslant 0, b \geqslant 0 \\
        \frac{a + b}{1 + (r - 1)ab - (r - 2)min(|a|, |b|)},& a \cdot b < 0 \\ 
        \frac{a + b + (r-2)ab}{1+(r-1)ab},& a \leqslant 0, b \leqslant 0 \\
      \end{array}
    \right.
  \]
\end{frame}

\subsection{Схемы Шортлиффа-Хамахера}
\begin{frame}
\frametitle{Схемы Шортлиффа-Хамахера}
С целью построения соответствующих функций $tms_r$ строится отображение $h_{1r}(x) = h_r^{-1}(h_1(x))$:
$$
  h_{1r}(x) = \sgn(x)\frac{1-(1-|x|)^\gamma}{1+(r-1)(1-|x|)^\gamma}
$$
$$
  h_{1r}^{-1}(x) = \sgn(x)\left(1 - \left(\frac{1 - |x|}{1+(r-1)|x|}\right)^\frac{1}{\gamma}\right)
$$

Такие отображения образуют семейство, зависящее от параметра $\gamma = \log_{\beta_1}\beta_2$.
\end{frame}

\begin{frame}
Таким образом, схема Шортлиффа-Хамахера - это преобразование схемы Шортлиффа с функцией $cmb_r$, в которой
$$ tms_r(A, B) = h_{1r}(h_{1r}^{-1}(A) \cdot h_{1r}^{-1}(B)) = \sgn(AB) \cdot \Gamma(\delta(A) + \delta(B) - \delta(A) \cdot \delta(B))$$
\begin{itemize}
\item{$\Gamma(x) = \frac{1 - x^\gamma}{1 + (r - 1)x^\gamma}$}
\item{$\delta(x) = \left(\frac{1 - |x|}{1+(r-1)|x|}\right)^\frac{1}{\gamma}$}
\end{itemize}
\end{frame}

\section*{}
\begin{frame}[plain,c]
\begin{center}
\Huge Спасибо за внимание!
\end{center}
\end{frame}

\end{document} % конец документа